% ================================================================
% МАКЕДОНСКИ МАТЕМАТИЧКИ УЧЕБНИК - LaTeX Template
% ================================================================

\documentclass[11pt,a4paper]{book}

% ----------------------------------------------------------------
% КИРИЛИЦА И ФОНТОВИ
% ----------------------------------------------------------------
\usepackage[T2A]{fontenc}
\usepackage[utf8]{inputenc}
\usepackage[macedonian,russian,english]{babel}
\usepackage{lmodern}  % Latin Modern фонтови

% ----------------------------------------------------------------
% МАТЕМАТИКА
% ----------------------------------------------------------------
\usepackage{amsmath}
\usepackage{amssymb}
\usepackage{amsthm}
\usepackage{mathtools}

% Дефинирање на теореми, дефиниции, примери
\theoremstyle{definition}
\newtheorem{definition}{Дефиниција}[section]
\newtheorem{example}{Пример}[section]
\newtheorem{exercise}{Вежба}[section]

\theoremstyle{plain}
\newtheorem{theorem}{Теорема}[section]
\newtheorem{lemma}{Лема}[section]

\theoremstyle{remark}
\newtheorem*{remark}{Забелешка}

% ----------------------------------------------------------------
% ГРАФИКА И ДИЈАГРАМИ
% ----------------------------------------------------------------
\usepackage{graphicx}
\graphicspath{{static_diagrams/}{animations/}}

% За TikZ (опционално)
\usepackage{tikz}
\usepackage{pgfplots}
\pgfplotsset{compat=1.18}

% ----------------------------------------------------------------
% LAYOUT И ФОРМАТИРАЊЕ
% ----------------------------------------------------------------
\usepackage[margin=2.5cm]{geometry}
\usepackage{setspace}
\onehalfspacing  % Поголем проредок за читливост

\usepackage{fancyhdr}
\pagestyle{fancy}
\fancyhf{}
\fancyhead[LE,RO]{\thepage}
\fancyhead[RE]{\nouppercase{\leftmark}}
\fancyhead[LO]{\nouppercase{\rightmark}}

% ----------------------------------------------------------------
% ДРУГИ КОРИСНИ ПАКЕТИ
% ----------------------------------------------------------------
\usepackage{xcolor}
\definecolor{mathblue}{RGB}{0,91,187}
\definecolor{examplegreen}{RGB}{0,128,0}

\usepackage{enumitem}
\usepackage{multicol}
\usepackage{hyperref}
\hypersetup{
    colorlinks=true,
    linkcolor=mathblue,
    citecolor=mathblue,
    urlcolor=mathblue,
    pdfauthor={Автор},
    pdftitle={Математика - 8 клас}
}

% ----------------------------------------------------------------
% CUSTOM COMMANDS
% ----------------------------------------------------------------
\newcommand{\N}{\mathbb{N}}  % Природни броеви
\newcommand{\Z}{\mathbb{Z}}  % Цели броеви
\newcommand{\Q}{\mathbb{Q}}  % Рационални броеви
\newcommand{\R}{\mathbb{R}}  % Реални броеви

% Боксови за примери
\usepackage[most]{tcolorbox}
\newtcolorbox{examplebox}{
    colback=examplegreen!5!white,
    colframe=examplegreen!75!black,
    title=Пример
}

\newtcolorbox{notebox}{
    colback=mathblue!5!white,
    colframe=mathblue!75!black,
    title=Забелешка
}

% ----------------------------------------------------------------
% МЕТАПОДАТОЦИ
% ----------------------------------------------------------------
\title{
    {\Huge Математика}\\[0.5cm]
    {\Large За 8 клас основно образование}
}
\author{Автор: Име Презиме}
\date{2026}

% ================================================================
% ПОЧЕТОК НА ДОКУМЕНТОТ
% ================================================================
\begin{document}

% ----------------------------------------------------------------
% НАСЛОВНА СТРАНА
% ----------------------------------------------------------------
\maketitle

\tableofcontents
\clearpage

% ================================================================
% ГЛАВА 1: РАЦИОНАЛНИ БРОЕВИ
% ================================================================
\chapter{Рационални броеви}

\section{Вовед}

Рационалните броеви се броеви кои можат да се запишат како дропка 
$\frac{p}{q}$, каде што $p \in \Z$ и $q \in \Z \setminus \{0\}$.

\begin{definition}[Рационален број]
Рационален број е број кој може да се претстави како количник на два 
цели броја, т.е. $r = \frac{p}{q}$, каде $p, q \in \Z$ и $q \neq 0$.
\end{definition}

% ----------------------------------------------------------------
% ВМЕТНУВАЊЕ НА ASYMPTOTE ДИЈАГРАМ
% ----------------------------------------------------------------
\subsection{Броевна оска}

\begin{figure}[h]
\centering
\includegraphics[width=0.8\textwidth]{number_line_example.pdf}
\caption{Броевна оска со рационални броеви}
\label{fig:number_line}
\end{figure}

\section{Операции со рационални броеви}

\subsection{Собирање и одземање}

\begin{examplebox}
Пресметај: $\frac{2}{3} + \frac{1}{4}$

\textbf{Решение:}
\begin{align*}
\frac{2}{3} + \frac{1}{4} &= \frac{2 \cdot 4}{3 \cdot 4} + \frac{1 \cdot 3}{4 \cdot 3} \\
                          &= \frac{8}{12} + \frac{3}{12} \\
                          &= \frac{11}{12}
\end{align*}
\end{examplebox}

\subsection{Множење и делење}

За множење на дропки важи:
\begin{equation}
\frac{a}{b} \cdot \frac{c}{d} = \frac{a \cdot c}{b \cdot d}
\end{equation}

% ----------------------------------------------------------------
% ВЕЖБИ
% ----------------------------------------------------------------
\section{Вежби}

\begin{exercise}
Пресметај:
\begin{multicols}{2}
\begin{enumerate}[label=\alph*)]
    \item $\frac{1}{2} + \frac{1}{3}$
    \item $\frac{3}{4} - \frac{1}{6}$
    \item $\frac{2}{5} \cdot \frac{3}{7}$
    \item $\frac{5}{8} : \frac{2}{3}$
\end{enumerate}
\end{multicols}
\end{exercise}

% ================================================================
% ГЛАВА 2: КВАДРАТНА ФУНКЦИЈА
% ================================================================
\chapter{Квадратна функција}

\section{Дефиниција и график}

\begin{definition}[Квадратна функција]
Функцијата $f(x) = ax^2 + bx + c$, каде $a \neq 0$, се нарекува 
квадратна функција.
\end{definition}

% ----------------------------------------------------------------
% ВМЕТНУВАЊЕ НА КОМПЛЕКСЕН ASYMPTOTE ДИЈАГРАМ
% ----------------------------------------------------------------
\begin{figure}[h]
\centering
\includegraphics[width=\textwidth]{kompleksen_primer.pdf}
\caption{График на функцијата $f(x) = x^2 - 4x + 3$}
\label{fig:parabola}
\end{figure}

Од Слика~\ref{fig:parabola} може да се види дека:
\begin{itemize}
    \item Темето на параболата е во точката $V(2, -1)$
    \item Нултите се $x_1 = 1$ и $x_2 = 3$
    \item Параболата пресекува ја $y$-оската во $(0, 3)$
\end{itemize}

\begin{notebox}
Параболата е симетрична во однос на правата $x = 2$ (оската на симетрија).
\end{notebox}

% ----------------------------------------------------------------
% INLINE TikZ ПРИМЕР
% ----------------------------------------------------------------
\subsection{Координатен систем}

Координатниот систем се состои од две меѓусебно нормални прави:

\begin{center}
\begin{tikzpicture}[scale=0.8]
    % Оски
    \draw[->] (-3,0) -- (3,0) node[right] {$x$};
    \draw[->] (0,-3) -- (0,3) node[above] {$y$};
    
    % Точка
    \fill[red] (2,1.5) circle (3pt);
    \draw[dashed] (2,0) -- (2,1.5) -- (0,1.5);
    \node[below] at (2,0) {$2$};
    \node[left] at (0,1.5) {$1.5$};
    \node[above right] at (2,1.5) {$P(2, 1.5)$};
\end{tikzpicture}
\end{center}

% ================================================================
% ДОПОЛНИТЕЛНИ РЕСУРСИ
% ================================================================
\chapter{Дополнителни ресурси}

\section{Листа на формули}

\begin{table}[h]
\centering
\begin{tabular}{|l|c|}
\hline
\textbf{Формула} & \textbf{Израз} \\
\hline
Площина на квадрат & $P = a^2$ \\
Площина на круг & $P = \pi r^2$ \\
Плоштина на триаголник & $P = \frac{a \cdot h}{2}$ \\
Питагорова теорема & $a^2 + b^2 = c^2$ \\
\hline
\end{tabular}
\caption{Основни геометриски формули}
\end{table}

% ================================================================
% КРАЈ
% ================================================================
\end{document}
